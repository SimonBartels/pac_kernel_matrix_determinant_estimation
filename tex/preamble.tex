% insights on package conflicts:
% cleveref should be loaded in the end
% autonum even later
% seems like thm-restate and autonum do not like each other => do not use align in restatable theorems
% ntheorem must come before any theorem declarations



\usepackage{microtype,marvosym} 
\usepackage{stmaryrd} % for \sslash 
\usepackage{amsmath,amssymb,graphicx}
%\usepackage[framed,numbered,autolinebreaks,useliterate]{mcode}
\usepackage{url}
%\usepackage{subcaption} % for subfigures
%\usepackage{amsthm} 
%\newtheorem{prop}{Proposition}
%\newtheorem*{thm*}{Theorem}
%\newtheorem*{corollary*}{Corollary}

\usepackage{afterpage}
\usepackage{nameref}
\usepackage{adjustbox}

% For use with matlab2tikz
\usepackage{pgfplots}
\pgfplotsset{compat=newest}
% \pgfplotsset{compat=1.15}
%\usetikzlibrary{external}

\pgfplotsset{
	every axis legend/.append style =
	{
		cells = { anchor = east },
		draw  = none
	},
}  
\makeatletter
\pgfplotsset{ 
	range frame/.style={
		tick align = outside,
		axis line style={opacity=0},
		after end axis/.code={
			\draw ({rel axis cs:0,0}-|{axis cs:\pgfplots@data@xmin,0}) -- ({rel axis cs:0,0}-|{axis cs:\pgfplots@data@xmax,0});
			\draw ({rel axis cs:0,0}|-{axis cs:0,\pgfplots@data@ymin}) -- ({rel axis cs:0,0}|-{axis cs:0,\pgfplots@data@ymax});
		}
	}
}
\makeatother

\pgfkeys{/pgfplots/mytuftestyle/.style={
		semithick,
		tick style={major tick length=4pt,semithick,black},
		separate axis lines,
		axis x line*=bottom,
		axis x line shift=5pt,
		xlabel shift=0pt,
		axis y line*=left,
		tick align = outside,
		axis y line shift=5pt,
		ylabel shift=0pt}}

%\usetikzlibrary{external}
\usepgfplotslibrary{external} 
\tikzsetexternalprefix{tikz_out/}
\tikzexternalize[mode=list and make]
% TIKZ
\usetikzlibrary{plotmarks}
\usepgfplotslibrary{groupplots}
\pgfplotsset{plot coordinates/math parser=false}       
\pgfkeys{/pgfplots/mystyle/.style={
		semithick,
		tick style={major tick length=4pt,semithick,gray},
		xtick align = inside,
		ytick align = inside,
		xlabel near ticks,
		ylabel near ticks,
		%every y tick scale label/.style={at={(rel axis cs:0,1)},anchor=south west,inner sep=1pt} % move scaling factors to the left of the plot
		%mark size = 0pt
%		every axis x label/.style={
%			at={(ticklabel cs:-0.1, -10)},anchor=near ticklabel
%		}
}}
\pgfkeys{/pgfplots/resultfigstyle/.style={
		semithick,
		tick style={major tick length=4pt,semithick,gray},
		xtick align = inside,
		xticklabel style = {align=center},
		ytick align = inside,
		xlabel near ticks,
		ylabel near ticks,
		every axis x label/.style={
				at={(ticklabel cs:-0.075, -25)},anchor=near ticklabel
			},
		every axis y label/.style={
				at={(ticklabel cs:1, -25)},anchor=near ticklabel
		},
		every tick label/.append style={font=\tiny}		
}}

% Check out version 30, if this trick fails ...
\newlength\figureheight
\newlength\figurewidth
\newlength\figheight
\newlength\figwidth



\usepackage{natbib}
\newcommand{\textcite}{\citet}
\newcommand{\parencite}{\citep}

\definecolor{lred}{RGB}{200,0,0}
\definecolor{dred}{RGB}{130,0,0} \definecolor{dblu}{RGB}{0,0,130}
\definecolor{dgre}{RGB}{0,130,0} \definecolor{dgra}{RGB}{50,50,50}
\definecolor{mgra}{RGB}{100,100,100}
\definecolor{lgra}{RGB}{220,220,220}
\definecolor{MPG}{RGB}{000,125,122}
\definecolor{mpg}{RGB}{000,125,122}
\definecolor{ora}{HTML}{FF9933}
\definecolor{newcode}{RGB}{220,220,220}


\definecolor{AMPurple}{HTML}{663366}
\definecolor{Burgundy}{HTML}{993333}
\definecolor{Coffee}{HTML}{7B6049}
\definecolor{ForestGreen}{HTML}{005826}
\definecolor{Lavender}{HTML}{6E6AB1}
\definecolor{PSLightBlue}{HTML}{7DA7D9}

\newcommand{\mpg}[1]{{\color{MPG} #1}}  
\newcommand{\gra}[1]{{\color{mgra} #1}}  
\newcommand{\dre}[1]{{\color{dred} #1}}  
\newcommand{\ora}[1]{{\color{ora} #1}}   
\newcommand{\blu}[1]{{\color{dblu} #1}}  

\usepackage{booktabs}

\newcommand{\g}{\,|\,} 
\newcommand{\de}{\partial}
\renewcommand{\d}{\:d} 
\newcommand{\eps}{\epsilon}
\newcommand{\dd}{\mathrm{d}}
\newcommand{\ddd}{\mathsf{d}}
\newcommand{\DD}{\mathsf{D}}
\newcommand{\Exp}{\mathbb{E}}
\newcommand{\Var}{\mathbb{V}}
\newcommand{\Ent}{\mathbb{H}}
\newcommand{\Cov}{\operatorname{cov}} 
\newcommand{\cov}{\operatorname{cov}} 
\newcommand{\erf}{\operatorname{erf}} 
\renewcommand{\H}{\mathcal{H}}
\newcommand{\K}{\mathcal{K}} 
\newcommand{\KL}{\text{KL}} 
\renewcommand{\Re}{\mathbb{R}}
\newcommand{\Co}{\mathbb{C}}
\newcommand{\one}{\mathbf{1}}
\newcommand{\const}{\text{const.}}
\newcommand{\diag}[1]{\operatorname{diag}\left[#1\right]}
\newcommand{\Dcal}{\mathcal{D}} 
\newcommand{\N}{\mathcal{N}} 
\newcommand{\W}{\mathcal{W}} 
\newcommand{\cS}{\mathcal{S}} 
\newcommand{\sT}{\mathsf{T}} 
\newcommand{\B}{\mathcal{B}} 
\renewcommand{\C}{\mathcal{C}}
\renewcommand{\L}{\mathcal{L}}
\newcommand{\Trans}{^{\intercal}} 
\newcommand{\iTrans}{^{-\intercal}} 
\newcommand{\pTrans}{^{+\intercal}} 
\newcommand{\argmin}{\operatorname*{arg\:min}}
\newcommand{\argmax}{\operatorname*{arg\:max}}
\newcommand{\arginf}{\operatorname*{arg\:inf}}
\newcommand{\argsup}{\operatorname*{arg\:sup}}
\renewcommand{\det}{\operatorname{det}}
\newcommand{\var}{\operatorname{var}}
\renewcommand{\=}{\operatorname*{=}}
\newcommand{\myexp}[1]{\exp{\left[ #1 \right] }}
\newcommand{\expmap}{\mathrm{Exp}}
\newcommand{\logmap}{\mathrm{Log}}

%\newcommand{\ostimes}{\otimes\hspace{-2.30046mm}-}
%\newcommand{\ostimes}{\otimes\hspace{-0.64031em}-}

% \usepackage{stackengine}
\newcommand{\ostimes}{\circledast}

% %%% all this just to load one document from MnSymbol. See also "FindSymbol.tex".
% \DeclareFontFamily{U} {MnSymbolC}{}
% \DeclareFontShape{U}{MnSymbolC}{m}{n}{
%   <-6> MnSymbolC5
%   <6-7> MnSymbolC6
%   <7-8> MnSymbolC7
%   <8-9> MnSymbolC8
%   <9-10> MnSymbolC9
%   <10-12> MnSymbolC10
%   <12-> MnSymbolC12}{}
% \DeclareFontShape{U}{MnSymbolC}{b}{n}{
%   <-6> MnSymbolC-Bold5
%   <6-7> MnSymbolC-Bold6
%   <7-8> MnSymbolC-Bold7
%   <8-9> MnSymbolC-Bold8
%   <9-10> MnSymbolC-Bold9
%   <10-12> MnSymbolC-Bold10
%   <12-> MnSymbolC-Bold12}{}

% \DeclareSymbolFont{MnSyC} {U} {MnSymbolC}{m}{n}

% \DeclareMathSymbol{\boxslash}{\mathrel}{MnSyC}{114}
% \DeclareMathSymbol{\boxbackslash}{\mathrel}{MnSyC}{115}

\newcommand{\utr}{\boxbackslash}
\newcommand{\ltr}{\boxslash}
\newcommand{\oodot}{\circledcirc}

\newcommand{\hl}[1]{\textcolor{blue}{#1}}

\newcommand{\q}{\quad}
\newcommand{\qq}{\qquad}
\newcommand{\qqq}{\quad\qquad}
\newcommand{\qqqq}{\qquad\qquad}

\renewcommand{\vec}{\boldsymbol} 
\newcommand{\mat}{\boldsymbol} 
\newcommand{\inv}[1]{{#1}^{-\!1}}
\newcommand{\trace}{\operatorname{trace}}
\newcommand{\vect}[1]{\overrightarrow{#1}}
\newcommand{\vest}[1]{\overrightharpoon{#1}}
\newcommand{\fun}[1]{\mathsf{#1}}
\newcommand{\logm}{\operatorname{Log}}
\newcommand{\expm}{\operatorname{Exp}}
\renewcommand{\O}{\mathcal{O}} 
\renewcommand{\G}{\mathcal{G}} 
\newcommand{\GP}{\mathcal{GP}}
\newcommand{\Id}{\vec{I}}
\newcommand{\zero}{\vec{0}}
\newcommand{\tr}[1]{\operatorname{tr}\left[#1\right]}
\newcommand{\rk}{\operatorname{rk}}
\newcommand{\II}{\mathbb{I}}
\renewcommand{\L}{\mathcal{L}}

\newcommand{\bm}{\boldsymbol{m}}
\newcommand{\bmu}{\boldsymbol{\mu}}
\newcommand{\bSigma}{\boldsymbol{\Sigma}}
\newcommand{\bPhi}{\boldsymbol{\Phi}}
\newcommand{\bphi}{\boldsymbol{\phi}}
\newcommand{\balpha}{\boldsymbol{\alpha}}
\newcommand{\bbeta}{\boldsymbol{\beta}}

\newcommand{\w}{\vec{w}}
\newcommand{\f}{\vec{f}}
\newcommand{\y}{\vec{y}}
\newcommand{\x}{\vec{x}}
\newcommand{\V}{\mathbb{V}}
\newcommand{\X}{\mathbb{X}}

\newcommand{\fS}{\mathsf{S}}
\newcommand{\fW}{\mathsf{W}}
\newcommand{\fB}{\mathsf{B}}
\newcommand{\fK}{\mathsf{K}}
\newcommand{\fH}{\mathsf{H}}
\newcommand{\fk}{\mathsf{k}}
\newcommand{\fh}{\mathsf{h}}
\newcommand{\fL}{\mathsf{L}}

\newcommand{\sA}{\boldsymbol{\mathsf{A}}}
\newcommand{\sB}{\boldsymbol{\mathsf{B}}}
\newcommand{\sC}{\boldsymbol{\mathsf{C}}}

\usepackage{colonequals}
\newcommand{\ce}{\colonequals}
\newcommand{\ec}{\equalscolon}

\usepackage{nicefrac}

\newcommand{\braket}[2]{\left\langle #1 , #2 \right\rangle}
\newcommand{\brakets}[3]{\left\langle #1 \, \middle| \, #2 \, \middle|\, #3 \right\rangle}

\usetikzlibrary{arrows,shapes,plotmarks}

\tikzset{>=stealth'} 
\tikzstyle{graphnode} = 
[circle,draw=black,minimum size=22pt,text centered,text
width=22pt,inner sep=0pt] 
\tikzstyle{var}   =[graphnode,fill=white]
\tikzstyle{obs}   =[graphnode,fill=black,text=white]
\tikzstyle{fac}   =[rectangle,draw=black,fill=black!25,minimum size=5pt]
\tikzstyle{facprior} =[rectangle,draw=black,fill=black,text=white,minimum size=5pt]
\tikzstyle{edge}  =[draw=white,double=black,thick,-]
\tikzstyle{prior} =[rectangle, draw=black, fill=black, minimum size=
5pt, inner sep=0pt]
\tikzstyle{dirprior} = [circle, draw=black, fill=black, minimum
size=5pt, inner sep=0pt]

% to avoid warnings, copy only two symbols from stmaryrd
\DeclareSymbolFont{stmry}{U}{stmry}{m}{n}
\DeclareMathSymbol\leftarrowtriangle\mathrel{stmry}{"5E}
\DeclareMathSymbol\rightarrowtriangle\mathrel{stmry}{"5F}
\renewcommand{\gets}{\operatorname*{\leftarrowtriangle}}
\renewcommand{\to}{\operatorname*{\rightarrowtriangle}}


\newcounter{PHcomment}
\newcommand{\PHcomment}[1]{%
	%note in the margin
	\refstepcounter{PHcomment}%
	{%
		%\setstretch{0.7}% spacing
		\todo[inline,color={MPG!20},size=\small]{%
			\textbf{Comment [PH \thePHcomment]:}~#1}%
}}

%\usepackage{tensor}
%% Left and right kernel derivatives
%\newcommand{\dk}{\tensor[^{\de}]{k}{}}
%\newcommand{\ddk}{\tensor[^{\de^2}]{k}{}}
%\newcommand{\ddkd}{\tensor[^{\de^2}]{k}{^{\de}}}
%\newcommand{\dddk}{\tensor[^{\de^3}]{k}{}}
%\newcommand{\dddkd}{\tensor[^{\de^3}]{k}{^{\de}}}
%\newcommand{\kd}{\tensor{k}{^{\de}}}
%\newcommand{\dkd}{\tensor[^{\de}]{k}{^{\de}}}
%\newcommand{\dy}{y^{\de}}

\usepackage{algorithm}
%\usepackage{algorithmic}
\usepackage{algpseudocode}
%\algrenewcommand{\algorithmiccomment}[1]{{\footnotesize \hfill$\rhd$ #1}}

%\usepackage[pdftex]{hyperref}

% Packages hyperref and algorithmic misbehave sometimes.  We can fix
% this with the following command.
%\newcommand{\theHalgorithm}{\arabic{algorithm}}

% Kronecker symbols
\makeatletter
\newcommand{\superimpose}[2]{%s
	{\ooalign{$#1\@firstoftwo#2$\cr\hfil$#1\@secondoftwo#2$\hfil\cr}}}
\makeatother
\newcommand{\sk}{\mathpalette\superimpose{{\otimes}{\ominus}}}
% \newcommand{\sk}{\circledast}
\newcommand{\kr}{\otimes}
\newcommand{\vecm}[1]{\operatorname{vec}\left(#1\right)} % vectorized matrix
\newcommand{\vecmtrans}[1]{\vecm{#1}\Trans} % vectorized matrix transposed (i.e. the vector -- not the matrix)
\newcommand{\tomat}[1]{\operatorname{mat}\left(#1\right)} % transform vector to matrix

\newcommand{\deriv}[2]{\frac{\partial #1}{\partial #2}}
\newcommand{\stack}[2]{\begin{bmatrix}
		#1 \\ #2
\end{bmatrix}}
\newcommand{\concat}[2]{\begin{bmatrix}
		#1 & #2
\end{bmatrix}}

\usepackage{epigraph}
\usepackage{amsfonts}
\usepackage{mdframed}
\renewcommand{\epsilon}{\varepsilon}

\newcommand{\eqcomment}[1]{\\*&\qquad(\text{{\small\emph{#1}}})\notag}

\newcommand{\norm}[1]{\|#1\|}


\newcommand{\pname}[1]{\emph{#1}} % style for proper names
\newcommand{\latin}[1]{\emph{#1}} % style for latin phrases
\newcommand{\adhoc}{\latin{ad hoc}}
\newcommand{\ie}{\latin{i.e.~}}
\newcommand{\eg}{\latin{e.g.~}}
\newcommand{\cf}{\latin{c.f.~}}

\renewcommand{\Re}{\mathbb{R}}
\newcommand{\reals}{\Re}
\newcommand{\sign}{\operatorname{sign}}

\renewcommand{\epsilon}{\varepsilon}

\newcommand{\abs}[1]{\left\lvert#1\right\rvert} % absolute value
\newcommand{\card}[1]{\# #1}% cardinality of a set
\newcommand{\smallabs}[1]{\lvert#1\rvert}
\renewcommand{\det}[1]{\mathrm{det}\left(#1\right)}% determinant
\newcommand{\pn}[1]{}
\newcommand{\Proba}{\mathbb{P}}
\newcommand{\proba}[1]{\Proba\left (#1\right )}% probability of an event
\newcommand{\marginnote}[1]{}
\newcommand{\todo}[1]{\textcolor{red}{#1}}
\newcommand{\integr}[1][\ ]{\int\limits_{#1} \! }
\newcommand{\dx}[1]{\mathrm{d}#1}


\renewcommand{\l}[1]{f_{#1}} %
\newcommand{\lBound}{C} % upper bound for \l{}_j-\Exp[\tilde{}_j\mid x_1, ..., x_{j-1}]
\newcommand{\me}[1][\tau]{\hat{\mu}_{#1}} % population mean estimate
\newcommand{\lDetn}[1]{D_{#1}} % log-determinant until of K_n
\newcommand{\Dhat}{\hat{D}_N} % the estimator for the log-determinant
\newcommand{\kj}[1][j]{\vec k_{#1}(\vec x)} % k_j(x) = [k(x, x_1), ..., k(x, x_j)]\Trans
\newcommand{\kjx}[2]{\vec k_{#1}(\vec #2)} % k_j(x) = [k(x, x_1), ..., k(x, x_j)]\Trans
\renewcommand{\L}{\mathcal{L}}
\renewcommand{\U}{\mathcal{U}}
\newcommand{\stopTimeOne}{\sign(\U_n)=\sign(\L_n)\neq 0}
\newcommand{\stopTimeTwo}{\frac{\U_n-\L_n}{2\min(\abs{\U_n}, \abs{\L_n})} \leq r}


\newcommand{\diff}{\est}
\newcommand{\splitL}{\sum_{j=\tau+1}^N l_j-\diff}
\newcommand{\splitR}{\epsilon_\tau}
\newcommand{\uBound}{\splitR}


\makeatletter
\def\th@plain{%
	\thm@notefont{}% same as heading font
	\itshape % body font
}
\def\th@definition{%
	\thm@notefont{}% same as heading font
	\normalfont % body font
}
\makeatother

% letter for performance metric
\newcommand{\metric}{m}

% commands used in the tikz files
\newcommand{\expconfig}[3]{#3}
\newcommand{\labely}{$\nicefrac{\U_n-\L_n}{2\min(\abs{\U_n}, \abs{\L_n})}$}
\newcommand{\labelx}{~}
\newcommand{\labelxu}{$d$}
\newcommand{\labelyu}{$m$}
\newcommand{\ticklabelsx}{true}
\newcommand{\ticklabelsy}{true}
\newcommand{\METRO}{METRO}
\newcommand{\PM}{PM2.5}
\newcommand{\TAMILNADU}{TAMILNADU}
\newcommand{\PROTEIN}{PROTEIN}
\newcommand{\BANK}{BANK}
\newcommand{\PUMADYN}{PUMADYN}
\newcommand{\overhead}{$5\%$ overhead}
\newcommand{\DefaultChol}{default$\quad$}
\newcommand{\StoppedChol}{probabilistic$\quad$}
\newcommand{\PivotedChol}{pivoting$\quad$}
\newcommand{\DiagonalPrecision}{diagonal precision$\quad$}
%\newcommand{\meantime}[1]{\textbf{\large default Cholesky mean time: #1 s}}
\newcommand{\meantime}[1]{\colorbox{white}{\textbf{\large mean time default Cholesky: #1 s}}}
\newcommand{\clearlegend}{}
\newcommand{\showlegend}{\renewcommand{\clearlegend}{}}
\newcommand{\hidelegend}{\renewcommand{\clearlegend}{\legend{}}}
% \runinfo: dataset, D, N, k, \theta, \ell, \delta 
\newcommand{\runinfo}[7]{$\ell=#6$}

% commands used in the tables
\newcommand{\transform}[1]{$\exp(#1)$}
\newcommand{\captionTransform}[1]{#1}

\newcommand{\augmentedStatement}[1]{\colorbox{newcode}{{#1}}}
\newcommand{\AState}[1]{\State \augmentedStatement{#1}} % additional state
%\newcommand{\AState}[1]{\State{#1}\hfill\COMMENT{\colorbox{dred}{NEW}}}
\newcommand{\indk}{j}
\newcommand{\indj}{k}
\newcommand{\indi}{i}

% caption text for the timing result figures
\newcommand{\captionText}[1]{
	relative execution times to compute the log-determinant using RBF (\textbf{left panel}) and OU (\textbf{right panel}) kernels on the #1 dataset for $\theta=1$, $\log\ell=-1, \dots, 3$ and $\delta=0.1$ for ten repetitions.
	The number next to one on the $y$-axis displays the absolute execution times of the default Cholesky.
	The solid, horizontal, orange line (\ref{leg:overhead}) visualizes the $105\%$ mark.
	The $x$-axis displays a desired absolute precision on the diagonal elements $d$ (top) and the average corresponding desired relative precision~$r$ (bottom) on the log-determinant.
}%

% seems for the following packages this is the best place: at the end of the preamble
\usepackage[capitalize]{cleveref} % MUST come after thm-restate but before theorem declarations
% it appears autonum must be loaded AFTER cleverref
\usepackage{autonum} % hide equation numbers if they are not referenced, disables equation* environments
